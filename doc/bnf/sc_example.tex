%% Copyright (c) 2008 Tasuku Hiraishi <hiraisi@kuis.kyoto-u.ac.jp>
%% All rights reserved.

%% Redistribution and use in source and binary forms, with or without
%% modification, are permitted provided that the following conditions
%% are met:
%% 1. Redistributions of source code must retain the above copyright
%%    notice, this list of conditions and the following disclaimer.
%% 2. Redistributions in binary form must reproduce the above copyright
%%    notice, this list of conditions and the following disclaimer in the
%%    documentation and/or other materials provided with the distribution.

%% THIS SOFTWARE IS PROVIDED BY THE AUTHOR ``AS IS'' AND
%% ANY EXPRESS OR IMPLIED WARRANTIES, INCLUDING, BUT NOT LIMITED TO, THE
%% IMPLIED WARRANTIES OF MERCHANTABILITY AND FITNESS FOR A PARTICULAR PURPOSE
%% ARE DISCLAIMED.  IN NO EVENT SHALL THE AUTHOR BE LIABLE
%% FOR ANY DIRECT, INDIRECT, INCIDENTAL, SPECIAL, EXEMPLARY, OR CONSEQUENTIAL
%% DAMAGES (INCLUDING, BUT NOT LIMITED TO, PROCUREMENT OF SUBSTITUTE GOODS
%% OR SERVICES; LOSS OF USE, DATA, OR PROFITS; OR BUSINESS INTERRUPTION)
%% HOWEVER CAUSED AND ON ANY THEORY OF LIABILITY, WHETHER IN CONTRACT, STRICT
%% LIABILITY, OR TORT (INCLUDING NEGLIGENCE OR OTHERWISE) ARISING IN ANY WAY
%% OUT OF THE USE OF THIS SOFTWARE, EVEN IF ADVISED OF THE POSSIBILITY OF
%% SUCH DAMAGE.

\documentstyle[11pt]{jarticle}

\def\|{\verb|} %|
  
\begin{document}
\begin{verbatim}
(def x int)
--> int x;

(extern x int)
--> extern int x;

(static y int (+ x 1)))
--> static int y=x+1;

(def (f x) (fn int int)
   (def a int) (def b int (+ x 1) (static c int 0)
   (= a (* b 2))
   (return (+ (++ a) (inc c)))))
--> int f (int x){
      int a;
      int b=x+1;
      static c=0;
      a = b*2;
      return ++a + c++;
    }

(def (f x) (fn int int))
--> int f(int x){}

(decl (f) (fn int int))
--> int f(int);

(extern-decl (f) (fn int int))
--> extern int f(int);

(def (struct s))
--> struct s{};

(decl (struct s))
--> struct s;



(def (struct s) 
   (def x int) (def y int) 
   (def d double))
--> struct s{
      int x,y;
      double d;
    }

(deftype xyd_t struct (def x int) (def y int) (d double))
(deftype xyd_ptr_d (ptr xyd_t))
--> typedef struct {
      int x, y;
      double d;
    } xyd_t;
    typedef xyd_t *xyd_ptr_t;

(deftype tagb struct 
     (def b0 unsigned-int) :bit 1
     (def b1 unsigned-int) :bit 1
     (def b2 unsigned-int) :bit 1 )
(def (union sb) (def s int) (def b tagb))
--> typedef struct {
        unsigned b0:1;
        unsigned b1:1;
        unsigned b2:1;
    } tagb;
    union sb{
      int s;
      tagb b;   
    };

(deftype tagc enum C0 C1)
(def tagc qq)
--> typedef enum { C0, C1 } tagc;
    tagc qq;



(def (enum abc) A B C)
--> enum abc { A, B, C };

(def a (array int 10))
--> int a[10];
\end{verbatim}
\|(def a (array int 5 2))| (\| (def a (array (array int 2) 5)) |)\\
\|--> int a[5][2];|
\begin{verbatim}
(def a1 (array int) (array 1 2 3))
--> int a1[]={1,2,3};

(def ar2 (array (array int 3)) (array (array 0 1 2) (array 3 4 5)))
--> int ar2[][3]={{0,1,2},{3,4,5}}
\end{verbatim}
\|(def (g) (fn int) (return (aref a x y)))|\\
(\| (def (g) (fn int) (return (aref (aref a x) y))) |)\\
\|--> int g(){return a[x][y];}|
\begin{verbatim}
(def (gg ff) (fn (ptr (fn (ptr void) int int))
                 (ptr (fn (ptr void) int int)))
     (return ff))  
--> void *(*gg(void *(*ff) (int, int))) (int, int) {
      return ff;
    }

(deftype gg-t (fn (ptr (fn (ptr void) int int))
                  (ptr (fn (ptr void) int int))))
--> typedef void *(*gg_t(void *(*) (int, int))) (int, int) ;

(def (f x) (fn int int) (register x) (return x))
--> int f(register x){ return x; }

(decl (f a b) (fn int char double va-arg))
--> int f(char a, double b, ... );

(def (f ld) (fn int long-double) attr: inline (return ld))
--> inline int f(long double ld){return ld;}
\end{verbatim}
\end{document}
